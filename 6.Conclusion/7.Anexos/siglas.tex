\chapter{Lineamientos generales}



De acuerdo a la Real Academia Espa\~nola (REA) se llama sigla tanto a la palabra formada por las iniciales de los t\'erminos que integran una denominaci\'on compleja, como a cada una de esas letras iniciales. Las siglas se utilizan para referirse de forma abreviada a organismos, instituciones, empresas, objetos, sistemas, asociaciones, etc.



\begin{enumerate}[1)] 

	\item \textbf{Tipos de siglas seg\'un su lectura}
		\begin{enumerate}[a)]
			\item {Hay siglas que se leen tal como se escriben, las cuales reciben tambi\'en el nombre de acr\'onimos: ONU, OTAN, ovni. Muchas de estas siglas acaban incorpor\'andose como sustantivos al l\'exico com\'un. Cuando una sigla est\'a compuesta solo por vocales, cada una de ellas se pronuncia de manera independiente y conserva su acento fon\'etico: OEA (\textit{Organizaci\'on de Estados Americanos}) se pronuncia [\'o-\'e-\'a].}

			\item {Hay siglas cuya forma impronunciable obliga a leerlas con deletreo: \textit{FBI} [\'efe-b\'e-\'i], \textit{DDT} [d\'e-d\'e-t\'e]. Integrando las vocales necesarias para su pronunciaci\'on, se crean a veces, a partir de estas siglas, nuevas palabras: \textit{elep\'e} (de \textit{LP}, sigla del ingl. \textit{long play} 'larga duraci\'on').}
			
			\item {Hay siglas que se leen combinando ambos m\'etodos: \textit{CD-ROM} [se-de-rr\'on, ze-de-rr\'en] (sigla del ingl. \textit{Compact Disc Read-Only Memory} 'disco compacto de solo lectura'). Tambi\'en en este caso pueden generarse palabras a partir de la sigla: \textit{cederr\'on}.}
			
	\end{enumerate}

	\item \textbf{Plural}. Aunque en la lengua oral tienden a tomar marca de plural ([oenej\'es] = 'organizaciones no gubernamentales'), son invariables en la escritura: las \textit{ONG}; por ello, cuando se quiere aludir a varios referentes es recomendable introducir la sigla con determinantes que indiquen pluralidad: \textit{Representantes de algunas/varias/numerosas ONG se reunieron en Madrid}. Debe evitarse el uso, copiado del ingl\'es, de realizar el plural de las siglas a\~nadiendo al final una \textit{s} min\'uscula, precedida o no de ap\'ostrofo: {\color{red} CD's, ONG's}.

	\item \textbf{G\'enero}. Las siglas adoptan el g\'enero de la palabra que constituye el n\'ucleo de la expresi\'on abreviada, que normalmente ocupa el primer lugar en la denominaci\'on: el FMI, por el \textit{Fondo Monetario Internacional}; la OEA, por la \textit{Organizaci\'on de Estados Americanos}; la UNESCO, por la \textit{United Nations Educational, Scientific and Cultural Organization} ('Organizaci\'on de Naciones Unidas para la Educaci\'on, la Ciencia y la Cultura'). Las siglas son una excepci\'on a la regla que obliga a utilizar la forma \textit{el} del art\'iculo cuando la palabra femenina que sigue comienza por /a/ t\'onica.

	\item \textbf{Ortograf\'ia}
		\begin{enumerate}[a)]
		\item {Las siglas se escriben hoy sin puntos ni espacios blancos de separaci\'on. S\'olo se escribe punto tras las letras que componen las siglas cuando van integradas en textos escritos enteramente en may\'usculas.}

		\item {Las siglas presentan normalmente en may\'uscula todas las letras que las componen (\textit{OCDE, DNI, ISO}) y, en ese caso, no llevan nunca tilde; as\'i, \textit{CIA} (del ingl. \textit{Central Intelligence Agency}) se escribe sin tilde, a pesar de pronunciarse [s\'ia, z\'ia], con un hiato que exigir\'ia acentuar gr\'aficamente la i. Las siglas que se pronuncian como se escriben, esto es, los acr\'onimos, se escriben solo con la inicial may\'uscula si se trata de nombres propios y tienen m\'as de cuatro letras: Unicef, Unesco; o con todas sus letras min\'usculas, si se trata de nombres comunes: uci, ovni, sida. Los acr\'onimos que se escriben con min\'usculas sí deben someterse a las reglas de acentuaci\'on gr\'afica: m\'odem.}

		\item {Si los d\'igrafos \textit{ch} y \textit{ll} forman parte de una sigla, va en may\'uscula el primer car\'acter y en min\'uscula el segundo: \textit{PCCh}, sigla de Partido Comunista de China.}

		\item {Se escriben en cursiva las siglas que corresponden a una denominaci\'on que debe aparecer en este tipo de letra cuando se escribe completa; esto ocurre, por ejemplo, con las siglas de t\'itulos de obras o de publicaciones peri\'odicas: \textit{DHLE}, sigla de Diccionario hist\'orico de la lengua espa\~nola; \textit{RFE}, sigla de Revista de Filolog\'ia Espa\~nola.}

		\item {Las siglas escritas en may\'usculas nunca deben dividirse con guion de final de l\'inea.}
	\end{enumerate}

	\item {Hispanizaci\'on de las siglas}. Siempre que sea posible, se hispanizar\'an las siglas: \textit{OTAN}, y no \textit{NATO}; \textit{ONU}, y no \textit{UNO}. Solo en casos de difusi\'on general de la sigla extranjera y dificultad para hispanizarla, o cuando se trate de nombres comerciales, se mantendr\'a la forma original: \textit{Unesco}, sigla de \textit{United Nations Educational, Scientific and Cultural Organization}; \textit{CD-ROM}, sigla de \textit{Compact Disc Read-Only Memory}; \textit{IBM}, sigla de \textit{International Business Machines}. Tampoco deben hispanizarse las siglas de realidades que se circunscriben a un pa\'is extranjero, sin correspondencia en el propio: \textit{IRA}, sigla de \textit{Irish Republic Army}. La primera vez que se emplea una sigla en un texto, y salvo que sea de difusi\'on tan generalizada que se sepa f\'acilmente interpretable por la inmensa mayor\'ia de los lectores, es conveniente poner a continuaci\'on, y entre par\'entesis, el nombre completo al que reemplaza y, si es una sigla extranjera, su traducci\'on o equivalencia: \textit{DEA} (\textit{Drug Enforcement Administration}, departamento estadounidense de lucha contra las drogas); o bien escribir primero la traducci\'on o equivalencia, poniendo despu\'es la sigla entre par\'entesis: la Uni\'on Nacional Africana de Zimbabue (\textit{ZANU}).
	
\end{enumerate}