\chapter{Introducci\'on}
	En la actualidad la calidad de los servicios entendida como la satisfacci\'on a las necesidades 
	y expectativas de las organizaciones debe estar en constante mejora, es decir, aumentar la eficiencia 
	de un sistema aplicando una pol\'itica de calidad, los objetivos de calidad, el an\'alisis de los datos y 
	las acciones correctivas y preventivas, identificando de qu\'e manera cada uno de estos procesos
	contribuyen a la mejora constante y optimizaci\'on, buscando simplificar las operaciones m\'as complejas, 
	mejorando los tiempos de respuesta en un Sistema de Gesti\'on de Calidad, por lo que es necesario realizar 
	revisiones constante de cada proceso, de esta manera podremos examinar con atenci\'on y cuidado para corregir 
	errores o para comprobar que funcione correctamente la organizaci\'on y asi poder tomar acciones preventivas y correctivas.\\

	Hoy en d\'ia las organizaciones cuentan con un Sistema de Gesti\'on de Calidad, sin tener la idea exacta de lo que esto significa, su concepto y los beneficios que se pueden tener cuando este se implementa adecuadamente con compromiso y liderazgo. Sistema de Gesti\'on de calidad significa planear, ejecutar y controlar las actividades realizadas dentro de una organizaci\'on, permitiendo con ello alcanzar el objetivo estipulado en la misi\'on y brindar servicios con altos est\'andares de calidad, los cuales son medidos a trav\'es de indicadores de satisfacci\'on de los usuarios. Como concepto tenemos que es una herramienta que permite planear, controlar y manejorar aquellos elementos de una organizac\'i\'on que influyen en el cumplimiento de los objetivos. Logrando como beneficio el buen funcionamiento de la organizaci\'on.\\

	El Tecnol\'ogico Nacional de M\'exico (TecNM) cuenta con un Sistema para  la Gesti\'on de la Calidad en el cual contiene las normativas para la realizaci\'on de buenas pr\'acticas y con ello brindar a los estudiantes el mejor servicio de educaci\'on.\\

	Por lo anterio dicho proyecto va orientado a la realizaci\'on de un Sistema Integral de Indicadores para el Sistema de Gesti\'on de Calidad del Instituto Tecnol\'ogico de Tl\'ahuac.\\

	Con la realizaci\'on de dicho Sistema de indicadores se pretende que los datos proporcionados por los indicadores de gesti\'on cuenten con la caracteristicas de relevancia, comprensibilidad, comparabilidad, oportunidad, consistencia y fiabilidad, asi mismo, reunan los rasgos de relevancia, verificabilidad, ausencia de sesgos, posibilidad de cuantificaci\'on, aceptabilidad institucional, factibilidad econ\'omica, comparabilidad y oportunidad. Dichos indicadores reuniran las siguientes caracter\'isticas.\\

	\begin{enumerate}
		\item  Relevante para la gesti\'on: Que aporte informaci\'on para informar, controlar, evaluar y tomar decisiones.
		\item  No ambiguo e inequ\'ivoco: Que no permita interpretaciones contrapuestas.
		\item  Pertinente: Que resulte adecuado a lo que se pretende medir.
		\item  Objetivo: Que no est\'e influenciado por factores externos.
	\end{enumerate}

	El Tecnol\'ogico Nacional de M\'exico (TecNM) cuenta con un Sistema para  la Gesti\'on de la Calidad en el cual contiene las normativas para la realizaci\'on de las mejoras pr\'acticas y con ello brindar a los estudiantes el mejor servicio de educaci\'on.\\

	Por lo anterio dicho proyecto va orientado a la realizaci\'on de un Sistema Integral de Indicadores para el Sistema de Gesti\'on de Calidad del Instituto Tecnol\'ogico de Tl\'ahuac.\\

	Los indicadores son herramientas necesarias para poder medir, y con ello, controlar los procesos con el objetivo de realizar una gesti\'on eficaz de los mismos.\\

	Seg\'un la AECA, los indicadores son "unidades de medida que permiten el seguimiento y la evaluaci\'on peri\'odica de las variables clave de una organizaci\'on, mediante su comparaci\'on con los correspondientes referentes internos y externos. Por su parte, G\'omez Rodriguez expone que "un indicador debe representar las magnitudes m\'as importantes del sistema as\'i como dar respuesta a todo tipo de variaciones del objeto de medici\'on". De manera m\'as concreta, y espec\'ifica para los indicadores de gesti\'on, De Forn se\~nala que estos indicadores tienen que permitir la medici\'on en un doble sentido: desde la vertiente de los resultados obtenidos y desde los recursos utilizados.\\

	Independientemente de la tipolog\'ia del indicador, hay que destacar que un indicador:\\

	\begin{enumerate}
		\item  Es una s\'intesis cuantitativa de uno o varios aspectos concretos de una determinada realidad.
		\item  Es una medida estad\'istica, de resumen, referida a la cantidad o magnitud de un conjunto de par\'ametros o atributos. Permite ubicar o clasificar las unidades de an\'alisis (personas, organizaciones, etc.) con respecto al concepto o conjunto de variables o atributos que se est\'an analizando.
		\item  Es una magnitud utilizada para medir o comparar los resultados efectivamente obtenidos, en la ejecuci\'on de un proyecto, programa o actividad.
		\item  Permite identificar las acciones cuyo efecto no se asemejan al est\'andar planteado.
	\end{enumerate}

	Con la realizaci\'on de dicho Sistema de indicadores se pretende que los datos proporcionados por los indicadores de gesti\'on cuenten con la caracteristicas de relevancia, comprensibilidad, comparabilidad, oportunidad, consistencia y fiabilidad, asi mismo, reunan los rasgos de relevancia, verificabilidad, ausencia de sesgos, posibilidad de cuantificaci\'on, aceptabilidad institucional, factibilidad econ\'omica, comparabilidad y oportunidad. Dichos indicadores reuniran las siguientes caracter\'isticas:\\

	\begin{enumerate}
		\item  Relevante para la gesti\'on: Que aporte informaci\'on para informar, controlar, evaluar y tomar decisiones.
		\item  No ambiguo e inequ\'ivoco: Que no permita interpretaciones contrapuestas.
		\item  Pertinente: Que resulte adecuado a lo que se pretende medir.
		\item  Objetivo: Que no est\'e influenciado por factores externos.
		\item  Sensible: Que capte tambi\'en los cambios peque\~nos.
		\item  Accesible: Que sea f\'acil de calcular y de interpretar.
	\end{enumerate}

	