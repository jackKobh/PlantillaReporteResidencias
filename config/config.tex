\renewcommand{\familydefault}{\sfdefault}%arial
	\begin{titlepage}
		\thispagestyle{empty}
			\begin{figure} [htbp]
				\begin{center}
					\includegraphics[width=16cm, height = 2cm]{logo}
				\end{center}
			\end{figure}
		%\vspace{1cm}
	%\vspace*{0in}
\begin{center}
\textsl{"LA ESENCIA DE LA GRANDEZA RADICA EN MIS RA�CES"}
\end{center}
\vspace*{1.5cm}


%\begin{Large}
\begin{center}
\textbf{INFORME T�CNICO DE RESIDENCIAS PROFESIONALES}\\
\end{center}
%\end{Large}
\vspace*{1.5cm}


\begin{center}
T�TULO DEL PROYECTO:
\end{center}
\begin{Large}
\begin{center}
\textbf{SISTEMA INTEGRAL DE INDICADORES PARA EL SGC DEL ITT }
\end{center}
\end{Large}
\vspace*{2cm} 


\begin{center}
\textbf{PRESENTA:}\\ URBANO BALLESTEROS RODRIGUEZ \\  VIANET VARELA CHIRINOS\\
\bigskip %salto de linea
\bigskip %salto de linea
\textbf{ASESOR:}\\ING. JUAN CARLOS CAMPOS CABELLO\\ING. EDSON JES�S ROSAS MART�NEZ
\end{center}
\vspace*{1cm}

\begin{figure}[htb]
\begin{flushleft}	
\includegraphics[width=6cm, height=2.5cm]{logo_carrera}
\end{flushleft}
\end{figure}
\vspace*{1cm}

\begin{flushright}
\textbf {Fecha de entrega:} \today
\textbf {.}
\end{flushright}
		\vfill
\end{titlepage}
\setcounter{page}{2}
\begin{center}
\end{center}
	\thispagestyle{empty}
	\newpage

	\newcommand{\bigrule}{\titlerule[0.5mm]}
	\titleformat{\chapter}[display] % cambiamos el formato de los cap�tulos
	{\bfseries\Huge} % por defecto se usar�n caracteres de tama�o \Huge en negrita
	{% contenido de la etiqueta
 	\titlerule % l�nea horizontal
 	\filleft % texto alineado a la derecha
	 	\Large\chaptertitlename\ % "Cap�tulo" o "Ap�ndice" en tama�o \Large en lugar de \Huge
	 	\Large\thechapter} % n�mero de cap�tulo en tama�o \Large
		{0mm} % espacio m�nimo entre etiqueta y cuerpo
	{\filleft} % texto del cuerpo alineado a la derecha
	[\vspace{0.5mm} \bigrule] % despu�s del cuerpo, dejar espacio vertical y trazar l�nea horizontal gruesa
	\pagestyle{fancy}
	\fancyhf{}
	%\fancyhead[LO]{\leftmark} % En las p�ginas impares, parte izquierda del encabezado, aparecer� el nombre de cap�tulo
	%\fancyhead[RE]{\rightmark} % En las p�ginas pares, parte derecha del encabezado, aparecer� el nombre de secci�n
	%\fancyhead[RO,LE]{\thepage} % N�meros de p�gina en las esquinas de los encabezados
	%2017-02-14
	% Por disposicion se modifico la estructura de la plantilla original para mostrar la numeracion en la parte inferior derecha.
	\fancyfoot[RE]{\thepage} % Páginas Pares numeradas a la derecha
	\fancyfoot[RO]{\thepage} % Páginas Impares numeradas a la izquierda

	\renewcommand{\chaptermark}[1]{\markboth{\textbf{\thechapter. #1}}{}} % Formato para el cap�tulo: N. Nombre
	\renewcommand{\sectionmark}[1]{\markright{\textbf{\thesection. #1}}} % Formato para la secci�n: N.M. Nombre

	\renewcommand{\headrulewidth}{0.6pt} % Ancho de la l�nea horizontal bajo el encabezado
	\renewcommand{\footrulewidth}{0.6pt} % Ancho de la l�nea horizontal sobre el pie (que en este ejemplo est� vac�o)
	


  % perzonalizar pie de pagina
\renewcommand\thefootnote{\textcolor{blue}{\bf\arabic{footnote}}}

% perzonalizar figuras
  \renewcommand\thefigure{\textcolor{blue}{\thechapter.\arabic{figure}}}
  \renewcommand{\figurename}{\textcolor{blue}{Figura}}
  
  \renewcommand\citeform[1]{\textcolor{blue}{#1}}%cambia el color al texto de las referencias
% perzonalizar tablas

  \renewcommand\thetable{\textcolor{blue}{\thechapter.\arabic{table}}}
  \renewcommand{\tablename}{\textcolor{blue}{Tabla}}
  
% perzonalizar el enumerate
\renewcommand{\labelenumi}{\textbf{\alph{enumi}$)$}} %letras
\renewcommand{\labelenumii}{\textbf{\arabic{enumii}$)$}} %numero arabigos
\renewcommand{\labelenumiii}{\textbf{\Roman{enumiii}.}} %numero ronanos
\renewcommand{\labelenumiv}{\textbf{\fnsymbol{enumiv}}}

\renewcommand{\labelitemi}{$\bullet$}%personaliza los itmes

	\setlength{\headheight}{1.5\headheight} % Aumenta la altura del encabezado en una vez y media

	%\marginsize{3cm}{2cm}{2.5cm}{2.5cm}% Margenes de la pagina

	\renewcommand\listfigurename{Lista de figuras}
	\renewcommand\listtablename{Lista de tablas}
	\renewcommand\listtablename{Lista de tablas}
	\renewcommand\contentsname{Tabla de contenido}
	\renewcommand\bibname{Bibliograf\'ia}

	