El framework utilizado en el servidor es .NET Framework de Microsoft el cual es un componente de software que puede ser añadido al sistema operativo Windows, este provee un extenso conjunto de soluciones predefinidas para necesidades generales de la programacion de aplicaciones y administra la ejecución de los programas escritos específicamente con la plataforma.
.NET Framework podría ser considerada una respuesta de Microsoft al creciente mercado de los negocios en entornos WEB, como competencia a la plataforma Java de Oracle Corporation y a los diversos Frameworks de desarrollo web usados en PHP. .NET propone ofrecer una manera rápida y económica, a la vez que segura y robusta para desarrollar aplicaciones permitiendo una integración rápida y ágil, permitiendo con esto acceso a la información desde cualquier tipo de dispositivo.
A continuación se muestra una breve descripción de las versiones de .NET Framework:
1.0
Lanzado en 2002 (Visual Studio .NET), Esta es la primera versión de .NET Framework, publicado el 13 de febrero de 2002 y disponible para Windows 98,ME,NT 4.0,2000 y XP. El soporte estándar de Microsoft para esta versión finaliza 10 de julio de 2007 y el soporte extendido terminó el 14 de julio de 2009, con la excepción de XP Media Center y Tablet PC ediciones. 
1.1
Lanzado en 2003 (Visual Studio 2003), El soporte integrado para teléfonos ASP.NET controles. Previamente disponible como un add-on para. NET Framework, que ahora forma parte del marco. Cambios en la seguridad - enable Windows Forms asambleas para ejecutar de manera semi-confianza de Internet, y permitir que código de acceso de seguridad en las aplicaciones ASP.NET. El soporte integrado para ODBC y bases de datos. Previamente disponible como un add-on para. NET Framework 1.0, que ahora forma parte del marco. NET Compact Framework - una versión del Framework para dispositivos pequeños. Protocolo de Internet versión 6 ( IPv6 ) de apoyo. Numerosos API cambia.
2.0
Lanzado en 2005 (Visual Studio .NET 2005)→ , con un nuevo CLR (para manejar los genéricos y tipos anulables) y los compiladores de C \# y VB 2 8.
El paquete redistribuible 2.0 se puede descargar de forma gratuita desde Microsoft, y fue publicado el 22 de enero de 2006. 2.0 El Software Development Kit (SDK) se puede descargar de forma gratuita desde Microsoft. Se incluye como parte de Visual Studio 2005 y Microsoft SQL Server 2005. Versión 2.0 sin ningún Service Pack es la última versión con soporte para Windows 98 y Windows Me. Versión 2.0 con Service Pack 2 es la última versión con soporte oficial para Windows 2000, aunque ha habido algunas soluciones no oficiales publicados en línea para utilizar un subconjunto de la funcionalidad de la Versión 3.5 en Windows 2000. [5] Versión 2.0 con Service Pack 2 requiere de Windows 2000 con SP4 además KB835732 o KB891861 actualización, Windows XP con Service Pack 2 o posterior y Windows Installer 3.1 (KB893803-v2) Se incluye con Windows Server 2003 R2 (no se instala por defecto).

3.0
Lanzado en 2006 (Expression Blend), este es sólo 2.0 además de nuevas bibliotecas: Windows Presentation Foundation, Windows Communication Foundation, Workflow Foundation y Cardspace. .NET Framework 3.0, anteriormente llamado WinFX, [ 6] fue lanzado el 21 de noviembre de 2006. Incluye un nuevo sistema de código administrado API que son una parte integral de Windows Vista y Windows Server 2008 sistemas operativos. También está disponible para Windows XP SP2 y Windows Server 2003 como descarga. No hay grandes cambios de arquitectura que se incluyen con esta versión;. NET Framework 3.0 utiliza el mismo Common Language Runtime.. (CLR) como NET Framework 2.0 [ 7 ]. A diferencia de las anteriores versiones principales NET no había NET Compact Framework versión hecha como. un homólogo de esta versión. La versión 3.0 de. NET Framework se incluye con Windows Vista. También se incluye con Windows Server 2008 como un componente opcional (desactivado por defecto).
Ha habido siete liberaciones significativas de. NET Framework, con exclusión de los service packs. El marco incluye los compiladores, tiempo de ejecución, y las bibliotecas. Además, hay otros perfiles como Silverlight que complican las cosas. A continuación la evolución de esta potente plataforma de trabajo:
3.0
.NET Framework 3.0 se compone de cuatro nuevos componentes principales:
Windows Presentation Foundation (WPF), anteriormente con nombre en código Avalon, una nueva interfaz de usuario del subsistema y API basadas en XML y gráficos vectoriales, que utiliza gráficos 3D equipo de hardware y Direct3D tecnologías. Ver WPF SDK para desarrolladores artículos y documentación sobre WPF.
Windows Communication Foundation (WCF), anteriormente con nombre en código Indigo, un sistema de mensajería orientada a servicios que permite a los programas interactúan localmente o remotamente similar a los servicios web.
Windows Workflow Foundation (WF) permite la construcción de la automatización de tareas y operaciones integradas con flujos de trabajo.
Windows CardSpace, anteriormente con nombre en código InfoCard, un componente de software que se almacena de forma segura la identidad digital de una persona y proporciona un sistema unificadode interfaz para la elección de la identidad para una transacción en particular, como acceder a un sitio web.
3.5
lanzado en 2007, esto es 3,0 más nuevas bibliotecas (algunos extras bibliotecas "base" como todo LINQ y TimeZoneInfo ) y nuevos (compiladores para C \# y VB 3.9)
Para los. NET Framework 3.5 SP1 también hay una nueva variante del. NET Framework, llamado ". NET Framework Client Profile", que a los 28 MB es significativamente menor que el marco completo y sólo instala componentes que son los más relevantes para escritorio de aplicaciones. Sin embargo, el cliente cantidades perfil para este tamaño sólo si se utiliza el programa de instalación en línea en Windows XP SP2 cuando se instalan ningún otro. NET Frameworks o utilizar de Windows Update. Cuando se utiliza el instalador fuera de línea o cualquier otro sistema operativo, el tamaño de la descarga es todavía 250 MB. 
4.0
Lanzado en 2010, lo que incluye un nuevo CLR (v4), nuevas bibliotecas y el DLR (Dynamic Language Runtime)
Clave se centra en esta versión son:
Extensiones paralelas para mejorar el apoyo para la computación en paralelo, que se dirigen a múltiples núcleos o distribuida sistemas. Para este fin, las tecnologías como PLINQ (Parallel LINQ ), una implementación paralela del motor LINQ, y tareas de la Biblioteca paralelo, que expone construcciones paralelas a través de llamadas a métodos., se incluyen. New Visual Basic. NET y C \# las características del lenguaje, como continuaciones de línea implícitas, distribución dinámica, parámetros con nombre y parámetros opcionales. Apoyo a los contratos de código. La inclusión de nuevos tipos de trabajar con aritmética de precisión arbitraria (System.Numerics.BigInteger) y los números complejos (System.Numerics.Complex).

4.5
Lanzado en 2012, esto permite el desarrollo WinRT en Windows 8, así como bibliotecas adicionales - con mucho más amplio async API Posibilidad de limitar la duración de la expresión regular del motor intentará resolver una expresión regular antes de que el tiempo de espera. Posibilidad de definir la cultura para un dominio de aplicación. El soporte de consola para Unicode ( UTF-16 codificación). Soporte para versiones de ordenamiento cadena cultural y comparación de datos. Un mejor rendimiento al recuperar los recursos. Zip mejoras de compresión para reducir el tamaño de un archivo comprimido. Posibilidad de personalizar un marco de reflexión para anular default reflexión comportamiento a través de la CustomReflectionContext clase.
Managed Extensibility Framework (MEF) [ fuente de edición 
Apoyo a genéricos tipos.
Convenio basado en modelo de programación que permite crear piezas basado en las convenciones de nombres en lugar de los atributos.
Múltiples ámbitos.
Operaciones asíncronas 
En el. NET Framework 4.5, se han añadido nuevas características asincrónicas a lenguajes C \# y Visual Basic. Estas características añaden un modelo basado en tareas para la realización de operaciones asincrónicas.
ASP.NET
Apoyo a los nuevos HTML5 tipos de formularios. Soporte para carpetas de modelo en los formularios Web Forms.
