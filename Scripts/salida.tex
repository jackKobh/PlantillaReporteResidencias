\chapter{Fundamento te\'orico}

	En el tercer cap\'itulo se presentan las t\'ecnicas y herramientas utilizadas para adoptar una buena orientaci\'on que permita sustentar el desarrollo del Sistema Integral de indicadores mostrando una investigaci\'on detallada con todo lo necesario para entender cada fase del mismo.

	\section{Indicador}
		El primer concepto que nos debe quedar claro es saber que es un indicador, un indicador es un dato que nos ayuda a medir de manera objetiva la evoluci\'on de un sistema de gesti\'on, por consiguiente un sistema de indicadores son datos que nos brindan informaci\'on cualitativa o cuantitativa, que permiten seguir el desarrollo de un proceso y su evaluaci\'on.

	\section{Sistema de gest\'ion}
		Es una herramienta que permite a una organizaci\'on planear, ejecutar y controlar las actividades realizadas en esta, aquellas que sean necesarias para el buen desarrollo de su misi\'on. El objetivo de los sistemas es aportar a la instituci\'on  un camino correcto para lograr el cumplimiento de las metas establecidas. \\
	
	Todo sistema de medici\'on debe satisfacer los siguientes objetivos:

	\begin{itemize}
		\item Comunicar la estrategia.
		\item Comunicar las metas.
		\item Identificar problemas y oportunidades.
		\item Diagnosticar problemas.
		\item Entender procesos.
		\item Definir responsabilidades
		\item Mejorar el control de la instituci\'on.
		\item Identificar iniciativas y acciones necesarias.
		\item Medir comportamientos.
		\item Facilitar la delegaci\'on en las personas.
		\item Integrar la compensaci\'on con la actuaci\'on. 
	\end{itemize}

	El Sistema Integral de Indicadores es un sistema de gesti\'on que organiza indicadores, permitiendo mediante su implementaci\'on obtener m\'ultiples beneficios a la comunidad del Instituto Tecnol\'ogico de Tl\'ahuac los cuales son:
	\begin{itemize}
		\item Mejorar\'a la imagen ante los alumnos que deseen ingresar a dicha instituci\'on.
		\item Se lograra brindar un servicio caracterizado por la tolerancia y la responsabilidad.
		\item Permitir\'a contar con informaci\'on \'util para la mejora continua de procesos.
		\item Disminuir\'a las demoras en la realizaci\'on de tr\'amites internos.
		\item Se realizar\'a una gesti\'on enfocada al beneficio del alumno.
		\item Se Lograr\'a el compromiso de los directivos con los objetivos organizacionales.
		\item Permitir\'a conocer la informaci\'on de los avances en cuanto a la evaluaci\'on de procesos para poder tomar decisiones  estrat\'egicas y permitir alcanzar los objetivos.
	\end{itemize}

	Con dicho Sistema se pretende abarcar cada uno de los departamentos estrat\'egicos que sustentan al Instituto Tecnol\'ogico de Tl\'ahuac permitiendo a los directivos tener a su alcance los datos de los indicadores en cada uno de ellos, buscando una vinculaci\'on entre estos para obtener estad\'isticas correctas y precisas. Los departamentos abordados se muestran en la tabla \ref{tabDeptos}.

\begin{table}[h]
\centering
\caption{Tabla de departamentos organizadas por subdirecciones}
\label{tabDeptos}
\begin{tabular}{@{}cc@{}}
\toprule
\multirow{8}{*}{SUBDIRECCION ACAD\'EMICA}                                                                                   & CIENCIAS B\'ASICAS                                                                      \\ \cmidrule(l){2-2} 
& \multicolumn{1}{c}{CIENCIAS DE LA TIERRA}                                            \\ \cmidrule(l){2-2} 
& \multicolumn{1}{c}{ELECTRICA Y ELECTRONICA}                                          \\ \cmidrule(l){2-2} 
& \multicolumn{1}{c}{SISTEMAS Y COMPUTACION}                                           \\ \cmidrule(l){2-2} 
& \multicolumn{1}{c}{CIENCIAS ECON\'OMICO-ADMINISTRATIVAS}                               \\ \cmidrule(l){2-2} 
& \multicolumn{1}{c}{METAL-MEC\'ANICA}                                                   \\ \cmidrule(l){2-2} 
& \multicolumn{1}{c}{DESARROLLO ACAD\'EMICO}                                             \\ \cmidrule(l){2-2} 
& \multicolumn{1}{c}{DIVISI\'ON DE ESTUDIOS PROFESIONALES}                               \\ \midrule
\multicolumn{1}{c}{\multirow{5}{*}{SUBDIRECCI\'ON ADMINISTRATIVA}}                                                        & \multicolumn{1}{c}{RECURSOS HUMANOS}                                                 \\ \cmidrule(l){2-2} 
\multicolumn{1}{c}{}                                                                                                    & \multicolumn{1}{c}{RECURSOS FINANCIEROS}                                             \\ \cmidrule(l){2-2} 
\multicolumn{1}{c}{}                                                                                                    & \multicolumn{1}{c}{RECURSOS MATERIALES Y DE SERVICIOS}                               \\ \cmidrule(l){2-2} 
\multicolumn{1}{c}{}                                                                                                    & \multicolumn{1}{c}{CENTRO DE COMPUTO}                                                \\ \cmidrule(l){2-2} 
\multicolumn{1}{c}{}                                                                                                    & \multicolumn{1}{c}{MANTENIMIENTO Y EQUIPO}                                           \\ \midrule
\multicolumn{1}{c}{\multirow{6}{*}{\begin{tabular}[c]{@{}c@{}}SUBDIRECCI\'ON DE PLANEACI\'ON\\ Y VINCULACI\'ON\end{tabular}}} & \multicolumn{1}{c}{GESTI\'ON TECNOL\'OGICA Y VINCULACI\'ON}                                \\ \cmidrule(l){2-2} 
\multicolumn{1}{c}{}                                                                                                    & \multicolumn{1}{c}{COMUNICACI\'ON Y DIFUSI\'ON}                                          \\ \cmidrule(l){2-2} 
\multicolumn{1}{c}{}                                                                                                    & \multicolumn{1}{c}{ACTIVIDADES EXTRAESCOLARES}                                       \\ \cmidrule(l){2-2} 
\multicolumn{1}{c}{}                                                                                                    & \multicolumn{1}{c}{SERVICIOS ESCOLARES}                                              \\ \cmidrule(l){2-2} 
\multicolumn{1}{c}{}                                                                                                    & \multicolumn{1}{c}{CENTRO DE INFORMACI\'ON}                                            \\ \cmidrule(l){2-2} 
\multicolumn{1}{c}{}                                                                                                    & \begin{tabular}[c]{@{}c@{}}PLANEACI\'ON, PROGRAMACI\'ON \\ Y PRESUPUESTACI\'ON\end{tabular} \\ \bottomrule
\end{tabular}
\end{table}

Los departamentos antes mencionados se encuentran detallados en el sitio oficial del Instituto Tecnol\'ogico de Tl\'ahuac en la seccion de departamentos.\\

Para el desarrollo del Sistema Integral de Indicadores se realiz\'o una investigaci\'on acerca de las tecnolog\'ias que se utilizar\'ian. Estas se encuentran clasificadas en dos partes importantes en el desarrollo web, la parte del cliente y la parte del servidor.\\

Las tecnolog\'ias que se encuentran en la parte del cliente son todas aquellas que son ejecutadas por el navegador web como son Javascript, HTML y CSS.\\

Las tecnolog\'ias del lado del servidor son todas aquellas en las cuales su ejecuci\'on se encuentra del lado del servidor como son C\#, VB, Java, Python y PHP. Estos lenguajes tienen la capacidad de comunicarse directamente con la base de datos, y con esto realizar los procesos necesarios para formar el resultado solicitado por el cliente.\\ 	

Dentro de los lenguajes de servidor se encuentran los lenguajes de consulta de base de datos, los cuales como su nombre lo dice sirven para manipular la informaci\'on de una base de datos. El lenguaje de manipulaci\'on de datos que se utiliza en los diferentes motores de base de datos es SQL.\\ \\
A continuaci\'on se definir\'an las tecnolog\'ias del cliente utilizadas para el desarrollo del sistema.

\section{HTML Y HTML 5}

Defini\'endolo de forma sencilla, \textsl{"HTML es lo que se utiliza para crear todas las p\'aginas web de Internet"}. M\'as concretamente, HTML es el lenguaje con el que se \textsl{"escriben"} la mayor\'ia de p\'aginas web.\\

Los dise\~nadores utilizan el lenguaje HTML para crear sus p\'aginas web, los programas que utilizan los dise\~nadores generan p\'aginas escritas en HTML y los navegadores que utilizamos los usuarios muestran las p\'aginas web despu\'es de leer su contenido HTML.\\

Aunque HTML es un lenguaje que utilizan los ordenadores y los programas de dise\~no, es muy f\'acil de aprender y escribir por parte de las personas.\\

El lenguaje HTML es un est\'andar reconocido en todo el mundo y cuyas normas define un organismo sin \'animo de lucro llamado World Wide Web Consortium, m\'as conocido como W3C. Como se trata de un est\'andar reconocido por todas las empresas relacionadas con el mundo de Internet, una misma p\'agina HTML se visualiza de forma muy similar en cualquier navegador de cualquier sistema operativo.\\

El propio W3C define el lenguaje HTML como \textsl{"un lenguaje reconocido universalmente y que permite publicar informaci\'on de forma global"}. Desde su creaci\'on, el lenguaje HTML ha pasado de ser un lenguaje utilizado exclusivamente para crear documentos electr\'onicos a ser un lenguaje que se utiliza en muchas aplicaciones electr\'onicas como buscadores, tiendas online y banca electr\'onica.\\

El origen de HTML se remonta al a\~no de 1980, cuando el f\'isico Tim Berners-Lee, trabajador del CERN propuso un nuevo sistema de hipertexto para compartir documentos.\\

Los sistemas de hipertexto hab\'ian sido desarrollados a\~nos antes. En el \'ambito de la inform\'atica el hipertexto permit\'ia que los usuarios accedieran a la informaci\'on relacionada con los documentos electr\'onicos que estaba visualizando. De cierta manera, los primitivos sistemas de hipertexto podr\'ian asimilarse a los enlaces de las p\'aginas web actuales.\\

Con el tiempo este sistema de hipertexto fue evolucionando convirti\'endose en el lenguaje de marcado m\'as utilizado en la actualidad, siendo su \'ultima versi\'on la conocida HTML5.\\

HTML5 establece una serie de nuevos elementos y atributos que reflejan el uso t\'ipico de los sitios web modernos, permitiendo concentrar b\'asicamente tres caracter\'isticas:

\begin{itemize}
	\item Estructura.
	\item Estilo.
	\item Funcionalidad.
\end{itemize}

Aunque oficialmente no fue declarado, HTML5 es considerado como la combinaci\'on de HTML, CSS y Javascript ya que estas tecnolog\'ias son altamente dependientes y act\'uan como una sola unidad organizada bajo  la especificaci\'on de HTML5. HTML est\'a a cargo de la estructura, CSS presenta esta estructura y Javascript se encarga de darle funcionalidad.

\section{CSS}

Como ya se mencion\'o anteriormente el conjunto de tecnolog\'ias que trabajan en conjunto con HTML para formar HTML5 no solamente incluye los nuevos elementos que permiten que la definici\'on de la estructura de un sistema se mas f\'acil, si no tambi\'en es de vital importancia hablar de la parte que permite darle una buena presentaci\'on.\\
Oficialmente CSS no tiene nada que ver con HTML5 ya que es un complemento desarrollado para superar las limitaciones y reducir la complejidad de HTML. En un principio HTML prove\'ia de atributos que permit\'ian definir estilos esenciales para cada elemento, pero a medida que el lenguaje evoluciono, la escritura de c\'odigos se volvi\'o compleja  y no pudo satisfacer las necesidades demandadas por los dise\~nadores, en consecuencia, CSS pronto fue adoptado gracias a que este permite separar la estructura de la presentaci\'on. Desde entonces CSS se ha convertido en uno de los lenguajes m\'as importantes de la actualidad enfocado en las necesidades de los dise\~nadores.\\
En la versi\'on 3 de CSS es considerado en el desarrollo de HTML5 , debido a esto la integraci\'on entre ambos lenguajes es vital para el desarrollo web  y es la raz\'on por la que cada que se habla de HTML5 tambi\'en se hace referencia a CSS3 aunque oficialmente se trate de dos tecnolog\'ias completamente separadas.
En estos momentos las nuevas caracter\'isticas que se incorporan a CSS3 se han ido incorporando a los navegadores web al igual que las caracter\'isticas de HTML5.

\section{Javascript}

Javascript es un lenguaje interpretado para m\'ultiples prop\'ositos pero actualmente solamente es considerado como un complemento. Dentro de las innovaciones  que tiene Javascript en la actualidad es el desarrollo de nuevos motores de interpretaci\'on, estos fueron creados para acelerar el procesamiento de c\'odigo. La clave de los motores m\'as exitosos es transformar el c\'odigo Javascript en c\'odigo m\'aquina para lograr velocidades de ejecuci\'on similares a las de una aplicaci\'on de escritorio. Con estas mejoras se pudo superar algunas limitaciones de rendimiento.\\
Para explotar al m\'aximo a Javascript se expandi\'o en relaci\'on con la portabilidad e integraci\'on. A su vez los navegadores fueron incluyendo por defecto las nuevas funcionalidades del lenguaje dentro de las cuales destacan las APIs (Interfaz de Programaci\'on de Aplicaciones). Estas APIs son interfaces para librer\'ias incluidas en los navegadores las cuales tienen como objetivo un f\'acil acceso 


\section{Bootstrap}

Para comenzar a hablar de bootstrap es necesario citar algunas definiciones las cuales aclararan el panorama acerca de este.

	\subsection{Framework}

	El concepto framework se emplea en muchos \'ambitos del desarrollo de sistemas software, no solo en el \'ambito de aplicaciones Web. Podemos encontrar frameworks para el desarrollo de aplicaciones m\'edicas, de visi\'on por computador, para el desarrollo de juegos, y para cualquier \'ambito que pueda ocurr\'irsenos.\\

 	En general, con el t\'ermino framework, nos estamos refiriendo a una estructura software compuesta de componentes personalizables e intercambiables para el desarrollo de una aplicaci\'on. En otras palabras, un framework se puede considerar como una aplicaci\'on gen\'erica incompleta y configurable a la que podemos a\~nadirle las \'ultimas piezas para construir una aplicaci\'on concreta.
