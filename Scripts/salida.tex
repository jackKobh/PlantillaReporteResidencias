\chapter{Introducci\'on}
	En la actualidad la calidad de los servicios entendida como la satisfacci\'on a las necesidades 
	y expectativas de las organizaciones debe estar en constante mejora, es decir, aumentar la eficiencia 
	de un sistema aplicando una pol\'itica de calidad, los objetivos de calidad, el an\'alisis de los datos y 
	las acciones correctivas y preventivas, identificando de que manera cada uno de estos procesos
	contribuyen a la mejora constante y optimizaci\'on, buscando simplificar las operaciones m\'as complejas, 
	mejorando los tiempos de respuesta en un Sistema de Gesti\'on de Calidad, por lo que es necesario realizar 
	revisiones constante de cada proceso, de esta manera podremos examinar con atenci\'on y cuidado para corregir 
	errores o para comprobar que funcione correctamente la organizaci\'on y asi poder tomar acciones para perfeccionarnos\\

	Hoy en d\'ia las organizaciones cuentan con un Sistema de Gesti\\'on de Calidad, sin tener la idea exacta de lo que esto significa, su concepto y los beneficios que se pueden tener cuando este se implementa adecuadamente con compromiso y liderazgo. Sistema de Gesti\'on de calidad significa planear, ejecutar y controlar las actividades realizadas dentro de una organizaci\'on, permitiendo con ello alcanzar el objetivo estipulado en la misi\'on y brindar servicios con altos est\'andares de calidad, los cuales son medidos a trav\'es de indicadores de satisfacci\'on de los usuarios. Como concepto tenemos que es una herramienta que permite planear, controlar y manejorar aquellos elementos de una organizac\'i\'on que influyen en el cumplimiento de los objetivos. Logrando como beneficio el perfeccionamiento de la organizaci\\'on\\

	El Tecnol\'ogico Nacional de M\'exico (TecNM) cuenta con un Sistema para  la Gesti\\'on de la Calidad en el cual contiene las normativas para la realizaci\'on de las mejoras pr\'acticas y con ello brindar a los estudiantes el mejor servicio de educaci\'on\\

	Por lo anterio dicho proyecto va orientado a la realizaci\\'on de un Sistema Integral de Indicadores para el Sistema de Gesti\'on de Calidad del Instituto Tecnol\'ogico de Tl\'ahuac\\

	Con la realizaci\'on de dicho Sistema de indicadores se pretende que los datos proporcionados por los indicadores de gesti\'on cuenten con la caracteristicas de relevancia, comprensibilidad, comparabilidad, oportunidad, consistencia y fiabilidad, asi mismo, reunan los rasgos de relevancia, verificabilidad, ausencia de sesgos, posibilidad de cuantificaci\'on, aceptabilidad institucional, factibilidad econ\'omica, comparabilidad y oportunidad. Dichos indicadores reuniran las siguientes caracter\'isticas\\

	\begin{enumerate}
		\item  Relevante para la gesti\'o\'on: Que aporte informaci\'on para informar, controlar, evaluar y tomar decisiones.
		\item  No ambiguo e inequ\'ivoco: Que no permita interpretaciones contrapuestas.
		\item  Pertinente: Que resulte adecuado a lo que se pretende medir.
		\item  Objetivo: Que no est\'e' influenciado por factores externos.
=======
	I N T R O D U C C I \'O N\\

	En la actualudad la calidad de los servicios entendida como la satisfacci\'on a las necesidades y expectativas de las organizaciones debe estar en constante mejora, es decir, aumentar la eficiencia de un sistema aplicando una politica de calidad, los objetivos de calidad, el analisis de los datos y las acciones correctivas y preventivas, identificando de que manera cada uno de estos procesos contribuyen a la mejora constante y optimizaci\'on, buscando simplificar las operaciones m\'as complejas, mejorando los tiempos de respuesta en un Sistema de Gesti\'on de Calidad, por lo que es necesario realizar revisiones constante de cada proceso, de esta manera podremos examinar con atenci\'on y cuidado para corregir errores o para comprobar que funcione correctamente la organizaci\'on y asi poder tomar acciones para perfeccionarnos.\\

	Hoy en dia las organizaciones cuentan con un Sistema de Gesti\'on de Calidad, sin tener la idea exacta de lo que esto significa, su concepto y los beneficios que se pueden tener cuando este se implementa adecuadamente con compromiso y liderazgo. Sistema de Gesti\'on de calidad significa planear, ejecutar y controlar las actividades realizadas dentro de una organizaci\'on, permitiendo con ello alcanzar el objetivo estipulado en la misi\'on y brindar servicios con altos est\'andares de calidad, los cuales son medidos a trav\'es de indicadores de satisfacci\'on de los usuarios. Como concepto tenemos que es una herramienta que permite planear, controlar y manejorar aquellos elementos de una organizac\'i\'on que influyen en el cumplimiento de los objetivos. Logrando como beneficio el perfeccionamiento de la organizaci\'on.\\

	El Tecnol\'ogico Nacional de M\'exico (TecNM) cuenta con un Sistema para  la Gesti\'on de la Calidad en el cual contiene las normativas para la realizaci\'on de las mejoras pr\'acticas y con ello brindar a los estudiantes el mejor servicio de educaci\'on.\\

	Por lo anterio dicho proyecto va orientado a la realizaci\'on de un Sistema Integral de Indicadores para el Sistema de Gesti\'on de Calidad del Instituto Tecnol\'ogico de Tl\'ahuac.\\

	Los indicadores son herramientas necesarias para poder medir, y con ello, controlar los procesos con el objetivo de realizar una gesti\'on eficaz de los mismos.\\

	Seg\'un la AECA, los indicadores son "unidades de medida que permiten el seguimiento y la evaluaci\'on peri\'odica de las variables clave de una organizaci\'on, mediante su comparaci\'on con los correspondientes referentes internos y externos". Por su parte, G\'omez Rodriguez expone que "un indicador debe representar las magnitudes m\'as importantes del sistema as\'i como dar respuesta a todo tipo de variaciones del objeto de medici\'on". De manera m\'as concreta, y espec\'ifica para los indicadores de gesti\'on, De Forn se\~nala que estos indicadores tienen que permitir la medici\'on en un doble sentido: desde la vertiente de los resultados obtenidos y desde los recursos utilizados.\\

	Independientemente de la tipolog\'ia del indicador, hay que destacar que un indicador:\\

	\begin{enumerate}
		\item  Es una s\'intesis cuantitativa de uno o varios aspectos concretos de una determinada realidad.
		\item  Es una medida estad\'istica, de resumen, referida a la cantidad o magnitud de un conjunto de par\'ametros o atributos. Permite ubicar o clasificar las unidades de an\'alisis (personas, organizaciones, etc.) con respecto al concepto o conjunto de variables o atributos que se est\'an analizando.
		\item  Es una magnitud utilizada para medir o comparar los resultados efectivamente obtenidos, en la ejecuci\'on de un proyecto, programa o actividad.
		\item  Permite identificar las acciones cuyo efecto no se asemejan al est\'andar planteado.
	\end{enumerate}

	Con la realizaci\'on de dicho Sistema de indicadores se pretende que los datos proporcionados por los indicadores de gesti\'on cuenten con la caracteristicas de relevancia, comprensibilidad, comparabilidad, oportunidad, consistencia y fiabilidad, asi mismo, reunan los rasgos de relevancia, verificabilidad, ausencia de sesgos, posibilidad de cuantificaci\'on, aceptabilidad institucional, factibilidad econ\'omica, comparabilidad y oportunidad. Dichos indicadores reuniran las siguientes caracter\'isticas:\\

	\begin{enumerate}
		\item  Relevante para la gesti\'on: Que aporte informaci\'on para informar, controlar, evaluar y tomar decisiones.
		\item  No ambiguo e inequ\'ivoco: Que no permita interpretaciones contrapuestas.
		\item  Pertinente: Que resulte adecuado a lo que se pretende medir.
		\item  Objetivo: Que no est\'e influenciado por factores externos.
>>>>>>> origin/UrbanoBallesteros
		\item  Sensible: Que capte tambi\'en los cambios peque\~nos.
		\item  Accesible: Que sea f\'acil de calcular y de interpretar.
	\end{enumerate}




	Esta es la introducc\'i\'on de mi proyecto\\
	
	Una forma efectiva de escribir la introducci\'on es tratar de seguir
	el patr\'on de un tri\'angulo, esto significa que primero se comienza con afirmaciones generales
	relativas al tema del trabajo reportado en el informe y poco a poco se va precisando m\'as
	lo que se dice hasta llegar al meollo del trabajo descrito. Al final de este tri\'angulo se debe
	decir alguna frase o p\'arrafo que resuma el contenido del reporte.\\
	
	En t\'erminos pr\'acticos podr\'ia decirse que una introducci\'on obedece a la formulaci\'on de las siguientes preguntas que antes de hacerlas el lector; se ha hecho ya el propio investigado.
	\begin{enumerate}
		\item  �Cu\'al es el tema del trabajo?
		\item  �Por qu\'e se hace el trabajo?
		\item  �C\'omo est\'a pensado el trabajo?
		\item  �Cu\'al es el m\'etodo empleado en el trabajo?
		\item  �Cu\'ales son las limitaciones del trabajo?
	\end{enumerate}
	Se recomienda que la informaci\'on se haga una vez terminado el desarrollo del trabajo, a fin de que no se proponga metas que no se van a alcanzar. En este sentido, no debe confundirse la introducci\'on con el plan de trabajo que el investigador elabora al principio, antes de abordar el estudio de los temas propuestos.\\ 
	
	
	El \'ultimo p\'arrafo de la introducci\'on debe contener una descripci\'on muy corta, no m\'as de una frase, del contenido
	de cada uno de los cap\'itulos o secciones en los cuales est\'a dividido el trabajo. Tal y como se
	hizo en cap\'itulo de introducci\'on de esta gu\'ia. La numeraci\'on se registra a partir de la introducci\'on, 
	siguiendo la numeraci\'on subsiguiente.\\
	
	Algunos puntos que el residente debe cubrir en este cap\'itulo son:
\begin{enumerate}
	\item Qu\'e movi\'o al residente a investigar precisamente este tema o problema y no otro. 

	\item C\'omo abord\'o el tema o el problema (principios, criterios, m\'etodos y procedimientos).
	
  \item Una descripci\'on breve de cada una de las partes del informe desde la justificaci\'on hasta las conclusiones y 
				recomendaciones. 	
	
	\item En el caso de que por pol\'iticas de la empresa se exija confidencialidad en algunos aspectos del proyecto, esto 
				deber\'a ser mencionado en la introducci\'on.
\end{enumerate}
 
	Dependiendo de la extensi\'on del informe completo y de la complejidad, la longitud de
	la introducci\'on puede variar, de una p\'agina completa como m\'inimo a unas cinco o diez
	p\'aginas. La introducci\'on debe tener sustancia, es decir, su contenido debe transmitir lo
	m\'as importante del trabajo reportado en el informe. No debe entrarse en demasiado detalle,
	pero tampoco deben decirse vaguedades, sino aspectos bien concretos del trabajo. Muchas
	personas lo primero que leen son la introducci\'on y las conclusiones, para tener una buena
	idea del trabajo realizado, por lo tanto es importante redactar con cuidado
	estas dos secciones, pues son la primera impresi\'on que se va a llevar el lector del trabajo.
	En general, la forma como est\'a escrita esta gu\'ia debe servir tambi\'en de ejemplo de lo
	que es un reporte, c\'omo organizarlo y presentarlo.
	

	