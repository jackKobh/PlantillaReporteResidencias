\chapter{Fundamento te\'orico}

	En el tercer cap\'itulo se presentan las t\'ecnicas y herramientas utilizadas para adoptar una buena orientaci\'on que permita sustentar el desarrollo del Sistema Integral de indicadores mostrando una investigaci\'on detallada con todo lo necesario para entender cada fase del mismo.

	\section{Indicador}
		El primer concepto que nos debe quedar claro es saber que es un indicador, un indicador es un dato que nos ayuda a medir de manera objetiva la evoluci\'on de un sistema de gesti\'on, por consiguiente un sistema de indicadores son datos que nos brindan informaci\'on cualitativa o cuantitativa, que permiten seguir el desarrollo de un proceso y su evaluaci\'on.

	\section{Sistema de gest\'ion}
		Es una herramienta que permite a una organizaci\'on planear, ejecutar y controlar las actividades realizadas en esta, aquellas que sean necesarias para el buen desarrollo de su misi\'on. El objetivo de los sistemas es aportar a la instituci\'on  un camino correcto para lograr el cumplimiento de las metas establecidas. \\
	
	Todo sistema de medici\'on debe satisfacer los siguientes objetivos:

	\begin{itemize}
		\item Comunicar la estrategia.
		\item Comunicar las metas.
		\item Identificar problemas y oportunidades.
		\item Diagnosticar problemas.
		\item Entender procesos.
		\item Definir responsabilidades
		\item Mejorar el control de la instituci\'on.
		\item Identificar iniciativas y acciones necesarias.
		\item Medir comportamientos.
		\item Facilitar la delegaci\'on en las personas.
		\item Integrar la compensaci\'on con la actuaci\'on. 
	\end{itemize}

	El Sistema Integral de Indicadores es un sistema de gesti\'on que organiza indicadores, permitiendo mediante su implementaci\'on obtener m\'ultiples beneficios a la comunidad del Instituto Tecnol\'ogico de Tl\'ahuac los cuales son:
	\begin{itemize}
		\item Mejorar\'a la imagen ante los alumnos que deseen ingresar a dicha instituci\'on.
		\item Se lograra brindar un servicio caracterizado por la tolerancia y la responsabilidad.
		\item Permitir\'a contar con informaci\'on \'util para la mejora continua de procesos.
		\item Disminuir\'a las demoras en la realizaci\'on de tr\'amites internos.
		\item Se realizar\'a una gesti\'on enfocada al beneficio del alumno.
		\item Se Lograr\'a el compromiso de los directivos con los objetivos organizacionales.
		\item Permitir\'a conocer la informaci\'on de los avances en cuanto a la evaluaci\'on de procesos para poder tomar decisiones  estrat\'egicas y permitir alcanzar los objetivos.
	\end{itemize}

	Con dicho Sistema se pretende abarcar cada uno de los departamentos estrat\'egicos que sustentan al Instituto Tecnol\'ogico de Tl\'ahuac permitiendo a los directivos tener a su alcance los datos de los indicadores en cada uno de ellos, buscando una vinculaci\'on entre estos para obtener estad\'isticas correctas y precisas. \\
	
	Los departamentos abordados ser\'an:

	\begin{table}[htb]
		\centering
		\begin{tabular}{|c|c|c|}
			\hline 
				\multirow{4}{5cm}{SUBDIRECCI\'ON ADMINISTRATIVA} & RECURSOS FINANCIEROS  & \multirow{4}{5cm}{
					2. Determinar las necesidades de recursos humanos del instituto tecnol\'ogico y presentarlas a la Subdirecci\'on de Servicios Administrativos para lo procedente.\\
 
					3. Coordinar la operaci\'on de los procesos de selecci\'on, contrataci\'on y desarrollo de personal conforme a las normas y lineamientos establecidos.\\
					 
					4. Coordinar los procesos derivados de las acciones del pago de remuneraciones del personal del instituto tecnol\'ogico conforme a las normas y lineamientos establecidos.\\
					 
					5. Coordinar la realizaci\'on de investigaciones de nuevos m\'etodos, t\'ecnicas y procedimientos relativos a la administraci\'on de personal, as\'i como de los estudios de factibilidad para su aplicaci\'on.\\

				} \\ 
				\cline{2-2} & Prueba & \\
				\cline{2-2} & Prueba & \\
				\cline{2-2} & Prueba & \\
			\hline
				\multirow{5}{5cm}{Espa\~na} & Madrid & \multirow{5}{5cm}{Espa\~na} \\ 
				\cline{2-2} & Prueba & \\
				\cline{2-2} & Prueba & \\
				\cline{2-2} & Prueba & \\
				\cline{2-2} & Prueba & \\
			\hline
		\end{tabular}
		\caption{Fusionando celdas.}
		\label{tabla:fusionandoceldas}
	\end{table}
