%Plantilla basada en "Template for Masters / Doctoral Thesis" (plantilla disponible en writeLaTex) que subió LaTeXTemplates.com

\documentclass[11pt]{book}
\usepackage[paperwidth=17cm, paperheight=22.5cm, bottom=2.5cm, right=2.5cm]{geometry}
\usepackage{amssymb,amsmath,amsthm} %paquete para símbolo matemáticos
\usepackage[spanish]{babel}
\usepackage[utf8]{inputenc} %Paquete para escribir acentos y otros símbolos directamente
\usepackage{enumerate}
\usepackage{graphicx}
%\usepackage{subfig} %para poner subfiguras
\graphicspath{{Img/}} %En qué carpeta están las imágenes
\usepackage[nottoc]{tocbibind}
\usepackage[pdftex,
            pdfauthor={Urbano Ballesteros Rodríguez},
            pdftitle={Sistema Integral de Indicadores para el SGC del ITT},
            pdfsubject={SISTEMAS Y COMPUTACIÓN},
            pdfkeywords={PALABRAS CLAVE},
            pdfproducer={Latex con hyperref},
            pdfcreator={pdflatex}]{hyperref}



\begin{document}

%----------------------------------------------------------------------------------------
%	COMANDOS PERSONALIZADOS
%----------------------------------------------------------------------------------------

%SI TU TESIS TIENE TEOREMAS Y DEMOSTRACIONES, PUEDES DESCOMENTAR Y USAR LOS SIGUIENTES COMANDOS

%\renewcommand{\proofname}{Demostración}
%\providecommand{\norm}[1]{\lVert#1\rVert} %Provee el comando para producir una norma.
%\providecommand{\innp}[1]{\langle#1\rangle} 
%\newcommand{\seno}{\mathrm{sen}}
%\newcommand{\diff}{\mathrm{d}}

%\newtheorem{teo}{Teorema}[section] 
%\newtheorem{cor}[teo]{Corolario}
%\newtheorem{lem}[teo]{Lema}

%\theoremstyle{definition}
%\newtheorem{dfn}[teo]{Definición}

%\theoremstyle{remark}
%\newtheorem{obs}[teo]{Observación}

%\allowdisplaybreaks


%----------------------------------------------------------------------------------------
%	PORTADA
%----------------------------------------------------------------------------------------

\title{TÍTULO DE LA TESIS} %Con este nombre se guardará el proyecto en writeLaTex

\begin{titlepage}
\begin{center}

%\textsc{\Large Instituto Tecnológico Autónomo de México}\\[4em]

%Figura
\begin{figure}[h]
\begin{center}
\includegraphics[width=13cm, height=2.5cm]{encabezado.png}
\end{center}
\end{figure}

\vspace{2em}

SISTEMA INTEGRAL DE INDICADORES PARA EL SGC DEL ITT

\vspace{5em}

Reporte Técnico de Residencias Profesionales

\vspace{6em}

Elaborado por:

\vspace{0.5em}

Urbano Ballesteros Rodríguez

\vspace{2em}

Asesores:

\vspace{0.5em}

Ing. Juan Carlos Campos Cabello

\vspace{0.5em}

Ing. Edson Jesús Rosas Martínez



\end{center}

\vspace*{\fill}
\textsc{México, CDMX \hspace*{\fill} 2014}

\end{titlepage}


\frontmatter

%----------------------------------------------------------------------------------------
%	TABLA DE CONTENIDOS
%---------------------------------------------------------------------------------------

\tableofcontents

\addcontentsline{toc}{chapter}{Lista de figuras} % para que aparezca en el indice de contenidos
\listoffigures % indice de figuras

\addcontentsline{toc}{chapter}{Lista de tablas} % para que aparezca en el indice de contenidos
\listoftables % indice de tablas

%----------------------------------------------------------------------------------------
%	TESIS
%----------------------------------------------------------------------------------------
\mainmatter %empieza la numeración de las páginas
\pagestyle{headings}

%  Incluye los capítulos en el folder de capítulos

\chapter{Objetivo del documento}


El objetivo primordial de la creación de esta plantilla es dar a conocer el entorno de trabajo en latex y sus beneficios.
\\

Látex es un sistema de composición de documentos de alta calidad; que incluye características diseñadas para la producción de documentación técnica y científica. LaTeX es el estándar de facto para la comunicación y publicación de documentos científicos. \cite{Latex}
\\
La fàcilidad de latex para la realización de documentos es verdaderamente grande, ya que permite realizar desde revistas, libros, articulos, informes técnicos y diapositivas.
\\
Latex realiza la organización del contenido como tablas y fíguras, las cuales ya no es necesario preocuparse por acomodar los numeros de referencia y con esto ahorrar tiempo.
\\
Latex cuenta con un módulo el cual genera la bibliografia, almacenando todo en un solo archivo. Estas bibliografías pueden ser usadas durante todo el documento. La funcionalidad del modulo bibliografico de Latex permite incluso mostrar unicamente las bibliograias referenciadas en el documento, permitiendo dejar de lado las que no se usen.
\thispagestyle{empty}
\chapter{Ejemplos de tablas}

Presentaremos algunos ejemplos de tablas.

\thispagestyle{empty}
\include{Capitulos/Conclusiones}
\thispagestyle{empty}


%----------------------------------------------------------------------------------------
%	APÉNDICES
%----------------------------------------------------------------------------------------

\addtocontents{toc}{\vspace{2em}} % Agrega espacios en la toc

\appendix % Los siguientes capítulos son apéndices

%  Incluye los apéndices en el folder de apéndices

\include{Apendices/Ap}
\thispagestyle{empty}
%\include{Apendices/AppendixB}
%\include{Apendices/AppendixC}

\addtocontents{toc}{\vspace{2em}} % Agrega espacio en la toc


%----------------------------------------------------------------------------------------
%	BIBLIOGRAFÍA
\bibliographystyle{plain}
\bibliography{biblio}

\nocite{*}
%\bibliography{Bib/biblio}
---------------------------------------------------







\end{document}