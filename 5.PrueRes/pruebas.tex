\chapter{Pruebas y resultados}

En este cap�tulo se presentan algunos de los aspectos m�s relevantes para desarrollar la escritura de los resultados de aprendizaje asociados a una determinada  materia o asignatura. Se trata de definir qu� se entiende por resultados de aprendizaje, el desarrollo de herramientas para su redacci�n y algunas claves y recomendaciones, para valorar su calidad una vez escritos con el fin de poder revisarlos con profundidad.\\

Asimismo, tambi�n se desarrolla ampliamente una serie de recomendaciones para el desarrollo de pruebas  que sean eficaces en cuanto a su relaci�n con los resultados de aprendizaje fundamentalmente, pero tambi�n respecto a las modalidades y m�todos de ense�anza utilizados.\\


\section{Gu�a de vocabulario para la escritura del componente verbal de aprendizaje}

Encontrar las palabras adecuadas en la escritura de un resultado este puede resultar una tarea dif�cil, especialmente si tenemos presentes otros elementos del dise�o de un cap�tulo: competencias, intenciones u objetivos del asesor.\\
 
\subsection{Actividades que proporcionan evidencia de conocimiento}

Definir, describir, identificar, etiquetar, listar, nombrar, reproducir, declarar, recordar, seleccionar, declarar, se consciente de, extraer, organizar, escribir, reconocer, medir,  subrayar, repetir, relacionar, conocer, asociar.

\subsection{Actividades que proporcionan evidencia de comprensi�n}
 
Interpretar, traducir, estimar, justificar, comprender, convertir, clarificar, defender, distinguir, explicar, extender, generalizar, ejemplificar, dar ejemplos de, inferir, parafrasear, predecir, rescribir, resumir, discutir, ejecutar, reportar, presentar, identificar, ilustrar, indicar, encontrar, seleccionar, comprender, representar, formular, juzgar, contrastar, clasificar, expresar, comparar.

\subsection{Actividades que proporcionan evidencia de aplicaci�n} 

Aplicar, resolver, construir, demostrar, cambiar, calcular, descubrir, manipular, modificar, operar, predecir, preparar, producir, relacionar, mostrar, usar, dar ejemplos, ejemplificar, dibujar, seleccionar, explicar c�mo, encontrar, elegir, evaluar, practicar, operar, ilustrar, verificar.

\subsection{Actividades que proporcionan evidencia de an�lisis}
 
Reconocer, distinguir entre, evaluar, analizar, diferenciar, identificar, ilustrar c�mo, inferir, destacar, se�alar, relacionar, seleccionar, separar, dividir, subdividir, comparar, contrastar, justificar, resolver, examinar, concluir, criticar, cuestionar, diagnosticar, identificar, categorizar, elucidar.

\subsection{Actividades que proporcionan evidencia de s�ntesis}
 
Proponer, presentar, estructurar, integrar, formular, ense�ar, desarrollar, combinar, compilar, componer, crear, dise�ar, explicar, generar, modificar, organizar, planificar, reestructurar, reconstruir, relacionar, reorganizar, revisar, escribir, resumir, conectar, reportar, alterar, argumentar, ordenar, seleccionar, gestionar, generalizar, precisar, derivar, concluir, construir, engendrar, sintetizar, agrupar, sugerir, extender.

\subsection{Actividades que proporcionan evidencia de evaluaci�n}
 
Juzgar, evaluar, concluir, comparar, contrastar, describir c�mo, criticar, discriminar, justificar, defender, evaluar, valorar, determinar, elegir, cuestionar, puntuar.\\

Analizando los componentes del �ltimo ejemplo:\\

			\begin{center}
			Verbo + Contenido + Naturaleza
			\end{center}


\section{Consejos para la escritura de pruebas y resultados}

\begin{itemize}
	\item En cualquier tipo de prueba y resultado se necesita que exista alguna clase de declaraci�n, bien sobre lo que el estudiante har� 					bien una referencia de la calidad del trabajo que ser� clave en la tarea para alcanzar los criterios de �xito marcados en �ste.
	\item Las pruebas y resultados deben evaluar o relacionarse con el aprendizaje que se menciona en el resultado.
	\item Redactar un punto cr�tico, o umbral, en los resultados proporciona m�s detalle a la evaluaci�n y permite precisar que el aprendizaje se ha conseguido.
\end{itemize}




\section{Ejemplos de redacci�n de criterios de evaluaci�n}
Resultado de aprendizaje: Al finalizar este cap�tulo, se espera que el estudiante sea capaz de explicar y demostrar los principales resultados de un reporte de residencias profesionales. Algunos criterio para las pruebas y resultados son:\\

\begin{itemize}
	\item Las pruebas y resultados deben estar escritas en procesador de textos y debe tener una extensi�n de superior a 5 cuartillas sobre 				el tema proporcionado.
	\item El resultado debe relacionarse con su prueba, as� como con el t�tulo del proyecto, debe estar claramente escrito y estructurado, 					demostrar la contribuci�n de lecturas complementarias y reflexi�n propia.
	\item El estudiante debe ser capaz de explicar c�mo  los resultados de las pruebas demuestran estos rasgos y c�mo contribuyen a su 							efectividad global.
	\item El alumno demostrar� al menos tres ejemplos de refuerzo positivo para mejorar las conductas.
\end{itemize}
   
