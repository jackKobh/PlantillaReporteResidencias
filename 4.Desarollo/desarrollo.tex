\chapter{Desarollo}

	Dentro de este cap\'itulo se desglosar\'an las etapas que se siguieron en el desarrollo del SII, cada una de estas es parte de la metodolog\'ia de cascada que fue la que se eligi\'o para este desarrollo.

	\section{An\'alisis del sistema}
		Una de las actividades m\'as importantes del Instituto Tecnol\'ogico de Tl\'ahuac es el c\'alculo de indicadores, esto debido a que estos brindan informaci\'on para la mejora de los procesos. El Sistema Integral de Indicadores debe ser capaz de mostrar la informaci\'on requerida por los usuarios de una manera f\'acil, flexible y r\'apida.\\

		La informaci\'on que se requiere del sistema es el seguimiento de los procesos de los diferentes departamentos. Los departamentos a administrar por el sistema son los siguientes:

		\begin{itemize}
			\item Acad\'emico.
			\item Vinculaci\'on.
			\item Planeaci\'on.
			\item Administraci\'on de los recursos.
			\item Calidad.
		\end{itemize}

		Cada uno de ellos requiere obtener informaci\'on espec\'ifica la cual permitir\'a mejorar cada uno de sus procesos.

		El sistema ser\'a capaz de recalcular los indicadores con los cambios que se le presenten a lo largo del periodo declarado, permitiendo con esto tomas decisiones estrat\'egicas para mejorar en caso de que los indicadores sean bajos, o mantener las buenas pr\'acticas en caso de que los indicadores sean los esperados.\\

		Los reportes a configurar dentro del sistema contendr\'an un n\'umero indefinido de c\'alculos, es decir, el n\'umero de datos a calcular es din\'amico permitiendo con esto darle flexibilidad y adaptaci\'on a diferentes circunstancias.\\


		Se necesita que el sistema tenga las siguientes caracter\'isticas:

		\begin{itemize}
			\item \textbf{Configurable:} El sistema debe tener la capacidad de ser configurado de acuerdo a las necesidades de los usuarios.
			\item \textbf{Escalable:} El sistema debe de ser capaz de adaptarse a nuevas funcionalidades y nuevos m\'odulos, esto quiere decir que debe estar listo para que nuevos desarrolladores puedan usar como base este desarrollo para futuras adaptaciones.
			\item \textbf{Amigable:} Es importante que la presentaci\'on del sistema sea agradable para el usuario, debido a que esto mejora la experiencia del mismo permitiendo al usuario tener un ambiente de trabajo agradable.
		\end{itemize}
		
		Con estas tres caracter\'isticas el sistema debe ser capaz de calcular y mostrar las cifras de los indicadores por departamento.

	\section{An\'alisis de requisitos del software}
	\section{Dise\~no del sistema}
	\section{Dise\~no del programa}
	\section{Codificaci\'on}
	\section{Pruebas}
	\section{Implantaci\'on}
	\section{Mantenimiento}